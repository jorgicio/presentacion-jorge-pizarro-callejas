\documentclass[10pt]{beamer}

\usetheme[progressbar=frametitle]{metropolis}
\usepackage{appendixnumberbeamer}
\usepackage{lipsum}
\usepackage{amsmath}
\usepackage{amssymb}
\usepackage{booktabs}
\usepackage[scale=2]{ccicons}
\usepackage{pgfplots}
\usepgfplotslibrary{dateplot}
\usepackage{dirtytalk}
\usepackage{xspace}
\newcommand{\themename}{\textbf{\textsc{metropolis}}\xspace}
\definecolor{mpigreen}{HTML}{007977}
\setbeamercolor{frametitle}{bg=mpigreen}

\title{Presentación personal}

\author{Jorge Pizarro Callejas}
\institute{Ingeniero en Ejecución en Informática \\ Universidad Técnica Federico Santa María}
\titlegraphic{\hfill\includegraphics[height=1.5cm]{Logo_UTFSM.png}}

\begin{document}


\maketitle

\begin{frame}{¿Quién soy?}
\begin{itemize}
\item Jorge Rodrigo Dante Pizarro Callejas.
\item \texttt{jpizarrocallejas@gmail.com}
\item Vivo en Quilpué, Región de Valparaíso, Chile.
\item Titulado de Ing. Ejec. Informática en la Universidad Técnica Federico Santa María, Valparaíso, Chile.
\item Teléfono de contacto: \texttt{+56976247629}
\end{itemize}
\end{frame}

\begin{frame}{Habilidades técnicas y áreas de interés}
 \begin{itemize}
  \item Diseño web.
  \item Programación en backend y frontend.
  \item Conocimiento del stack de AWS (Cloudformation, Lambda, Cloud, EC2, S3, etc.)
  \item Desarrollo y consumo de APIs.
  \item Lenguajes de programación: JavaScript (Node.js), Java, Python, C/C++, Shell (Bash), PHP, HTML+CSS, Latex.
  \item Frameworks: Django, Bootstrap, Foundation, TailwindCSS, Angular, Vue.js
  \item Bases de datos: MySQL, PostgreSQL, Dynamo, Athena.
  \item Herramientas: VSCode, Git, NodeJS, Docker.
 \end{itemize}
\end{frame}

\begin{frame}{Breve resumen de mi experiencia laboral}
 Mi experiencia laboral empezó en 2011 cuando me integré a DelPuerto para hacer mi práctica laboral como diseñador web levantando sitios en WordPress y diseñando y modificando temas. Estuve hasta el 2014 cuando trabajé en la PUCV Campus Sausalito para un proyecto del programa Conecyt aplicado al aprendizaje de inglés, hecho con PHP, JS y MySQL. 
 En 2015, trabajé por 2 meses en Everis para diseñar el panel de usuario de Consorcio. 
 En 2016, volví a trabajar con Wordpress para una página para un centro kinesiológico.
 En 2019, trabajé durante el mes de diciembre en Kimun Ltda. como desarrollador frontend para el proyecto "Mundo de vegetales".
 En 2021, trabajé por más de 3 años en Waypoint (actualmente Tranciti) como ingeniero desarrollador junior, esta vez más del lado backend apoyándome con AWS como infraestructura en la nube, además de proyectos en Java y Node.js, y en algunos proyectos usé Angular para el frontend.
\end{frame}

\begin{frame}{Resolución de problemas}
En mis años como desarrollador, los problemas los resuelvo siguiendo un enfoque estructurado, que incluye análisis, definción, investigación, planificación, implementación y validación, los cuales varían dependiendo del problema en cuestión.
Para el caso de frontend, lo cual es requerido para el puesto, primero replico el problema en diferentes entornos y dispositivos. Luego con las DevTools de Chrome, realizo el debugging, analizo el flujo de datos y a partir de ahí empiezo a analizar e implementar la solución para el problema en cuestión. Por ejemplo, cuando he trabajado con frameworks de Node como Angular, levanto el sitio en un ambiente local para facilitar el debugging y ya encontrado el problema, implemento la solución, vuelvo a levantar y luego lo pruebo en otros dispositivos para verificar que haya estado solucionado. Luego se hacen pruebas, revisiones de pares y dependiendo de ello, se realiza el deploy a desarrollo, se revisa nuevamente y cuando todo esté operativo, se pasa a producción.
\end{frame}

\begin{frame}{Trabajos más relevantes para desarrollador frontend}
\textit{Valparaíso es un Cuento}, página promocional de dicho evento, hecho en 2011 en mi época en DelPuerto, Valparaíso.
Para ello, se editó un tema de WordPress con algunos cambios en un tema hijo. Además, se necesitó editar código para ello.

\includegraphics[scale=0.2]{trabajos/02.jpg}
\end{frame}

\begin{frame}{Trabajos más relevantes para desarrollador frontend}
\textit{Mil Tambores para Violeta}, página promocional de dicho evento, hecho en 2011 en mi época en DelPuerto, Valparaíso. Hecho con WordPress usando un tema modificado por mí.

\includegraphics[scale=0.2]{trabajos/03.jpg}
\end{frame}

\begin{frame}{Trabajos más relevantes para desarrollador frontend}
\textit{Prácticas Laborales, PUCV}, página del portal de prácticas laborales de la Pontificia Universidad Católica de Valparaíso, hecho en 2011 junto a DelPuerto.
Hecho en WordPress con un tema modificado.

\includegraphics[scale=0.2]{trabajos/04.jpg}
\end{frame}

\begin{frame}{Trabajos más relevantes para desarrollador frontend}
\textit{Panel de usuario, Consorcio}. Sitio frontend hecho con Bootstrap. 2015, en Everis.

\includegraphics[scale=1]{trabajos/06.jpg}
\end{frame}



\begin{frame}{Trabajos más relevantes para desarrollador frontend}
\textit{Sitio web personal}, trabajo en progreso. Hecho con Vue.js para el frontend, con TailwindCSS como framework de diseño.

\includegraphics[scale=0.2]{trabajos/websitepersonal.png}
\end{frame}
 
\begin{frame}{¿Más trabajos?}
Puedes visitar mis portafolios para más trabajos en:
 \begin{itemize}
  \item Behance: \url{https://behance.net/jorgepizarrocallejas}
  \item GitHub: \url{https://github.com/jorgicio}
 \end{itemize}
\end{frame}

\begin{frame}{Fin de la Presentación}
Muchas gracias por tomarse el tiempo de leer.
\end{frame}

\end{document}